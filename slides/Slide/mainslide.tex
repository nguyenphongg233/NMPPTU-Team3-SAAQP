\documentclass[aspectratio=169]{beamer}
\usepackage{algorithm}
\usepackage{algpseudocode}
\usepackage{tikz}
\usepackage{calc}
\usepackage{polyglossia}
\setmainlanguage{vietnamese}
\usepackage{amsthm,amsmath,amssymb}
\newtheorem{thm}{Định lý}
\newtheorem{defn}{Định nghĩa}
\renewcommand{\proofname}{\textbf{Chứng minh}}
\renewcommand{\qedsymbol}{$\blacksquare$}      
\usepackage{subcaption}
\setbeamertemplate{theorems}[numbered]

%% TYPESET
\makeatletter
\newcommand{\leqnomode}{\tagsleft@true}
\newcommand{\reqnomode}{\tagsleft@false}
\makeatother
\let\Mathcal\mathcal
\usepackage{dutchcal}
\newcommand{\norm}[1]{\left\lVert#1\right\rVert}
\newcommand{\abs}[1]{\left\lvert#1\right\rvert}
\newcommand{\gr}{\operatorname{gr}}
\newcommand{\rank}{\operatorname{rank}}
\newcommand{\toto}{\rightrightarrows}%
\newcommand{\F}{\Mathcal{F}}%
\newcommand{\K}{\mathbb{K}}%
\newcommand{\R}{\mathbb{R}}%
\def\C{\mathbb{C}}%
\newcommand{\dom}{\operatorname{dom}}%
% \newcommand{\ker}{\operatorname{ker}}% already defined
\newcommand{\im}{\operatorname{im}}%
\newcommand{\g}{\mathcal{g}}%
\global\long\def\MP{\text{mp}}%

\newcommand{\Image}[1]{%
	\sbox0{\includegraphics[height=0.65\paperheight]{#1}}%
	\ifdim\wd0 < \textwidth
	\includegraphics[height=0.65\paperheight]{#1}%
	\else
	\includegraphics[width=\textwidth]{#1}%
	\fi%
}

\newif\iffirsttoc
\firsttoctrue
\AtBeginSubsection[]
{
    \begin{frame}
        \frametitle{\textbf{\large Nội dung}}  
        \iffirsttoc
            \tableofcontents
            \global\firsttocfalse
        \else
            \tableofcontents[currentsection,
            subsectionstyle=show/shaded/hide,
            subsubsectionstyle=show/show/show/hide
            ]
        \fi
    \end{frame} 
}
\setbeamertemplate{caption}[numbered]
\usetheme{hust}
% \titlegraphic{\includegraphics[height=\logoheight]{figures/sami-v2.pdf}}
%\title{Seminar II:}
% \subtitle{\textbf{\LARGE Tìm bán kính điều khiển có cấu trúc của hệ chịu\\đa nhiễu sử dụng các toán tử tuyến tính đa trị}}
\title{\textbf{Self-adaptive algorithms for quasiconvex programming \\ and applications to machine learning }}

\author{Nhóm 3}
\date{Tháng 12, 2025}

\begin{document}
\begin{frame}[noframenumbering,Title]
	\maketitle
\end{frame}

\begin{frame}{\textbf{\large Thành viên nhóm}}
    \begin{large}   
        \begin{enumerate}
            \item Nguyễn Phong \hfill \makebox[4.5cm][l]{MSSV: 202400066}
            \item Nguyễn Đăng Long \hfill \makebox[4.5cm][l]{MSSV: 202400057}
            \item Phạm Gia Linh \hfill \makebox[4.5cm][l]{MSSV: 202416262}
            \item Nguyễn Tuấn Long \hfill \makebox[4.5cm][l]{MSSV: 202416269}
            \item Đoàn Đức Mạnh \hfill \makebox[4.5cm][l]{MSSV: 202400058}
            \item Trần Đình Nam \hfill \makebox[4.5cm][l]{MSSV: 202400063}
            \item Hoàng Thị Thu Phương \hfill \makebox[4.5cm][l]{MSSV: 202400068}
            \item Đặng Bảo Quân \hfill \makebox[4.5cm][l]{MSSV: 202416319}
            \item Phạm Thị Bích Phương \hfill \makebox[4.5cm][l]{MSSV: 202400069}
            \item Dương Thanh Minh \hfill \makebox[4.5cm][l]{MSSV: 20230047}  
            \item Nguyễn Hải Yến Nhi \hfill \makebox[4.5cm][l]{MSSV: 202400064}
            \item Lương Đức Mạnh \hfill \makebox[4.5cm][l]{MSSV: 20235152}
        \end{enumerate}
    \end{large}
\end{frame}


\section{1 Kiến thức cần biết}
\begin{frame}{\textbf{\large Nội dung phần Kiến thức cần biết}}
    \vspace{-1cm}
    \tableofcontents[
        currentsection 
    ]
\end{frame}

\subsection{1.1 Giả thiết và Bài toán}

\begin{frame}[t]{\textbf{\large 1.1 Giả thiết và Bài toán}}
    \begin{itemize}
        \item Giả sử $C$ là một tập hợp \textbf{khác rỗng}, \textbf{đóng} và \textbf{lồi} trong $\mathbb{R}^m$.
        \item Hàm $f : \mathbb{R}^m \to \mathbb{R}$ là một hàm khả vi trên một tập mở chứa $C$.
        \item Ánh xạ $\nabla f$ là liên tục $L$-Lipschitz, tức là tồn tại $L > 0$ sao cho:
        \[
            \|\nabla f (\mathbf{x}) - \nabla f (\mathbf{y})\| \le L\|\mathbf{x} - \mathbf{y}\|, \quad \forall \mathbf{x}, \mathbf{y} \in C
        \]
    \end{itemize}

    \pause
    \vspace{0.5cm} 
    
    \textbf{Bài toán tối ưu:}
    \[
        \min_{\mathbf{x} \in C} f(\mathbf{x}), \quad \hypertarget{target:OP}{OP(f,C)}
    \]
\end{frame}

\subsection{1.2 Phép chiếu lên tập lồi}

\begin{frame}[t]{\textbf{\large 1.2 Phép chiếu lên tập lồi}}
    \begin{itemize}
        \item Giả sử tập nghiệm của $OP(f, C)$ là \textbf{không rỗng}.
    \end{itemize}
    
    \vspace{0.3cm}
    
    Với $\mathbf{x} \in \mathbb{R}^m$, ký hiệu $P_C(\mathbf{x})$ là \textbf{phép chiếu} của $\mathbf{x}$ lên $C$:
    \[
        P_C(\mathbf{x}) := \arg\min\{\|\mathbf{z} - \mathbf{x}\| : \mathbf{z} \in C\}.
    \]

    \pause
    \vspace{0.5cm}

    \begin{block}{Mệnh đề 1 (Bauschke và Combettes 2011)}
        Các khẳng định sau là đúng:
        \begin{enumerate}
            \item $\|P_C(\mathbf{x}) - P_C(\mathbf{y})\| \le \|\mathbf{x} - \mathbf{y}\|$ với mọi $\mathbf{x}, \mathbf{y} \in \mathbb{R}^m$.
            \item $(\mathbf{y} - P_C(\mathbf{x}))^T(\mathbf{x} - P_C(\mathbf{x})) \le 0$ với mọi $\mathbf{x} \in \mathbb{R}^m, \mathbf{y} \in C$.
        \end{enumerate}
    \end{block}
\end{frame}

\subsection{1.3 Các định nghĩa và tính chất}

\begin{frame}[t]{\textbf{\large 1.3.1 Các định nghĩa về tính lồi}}
    \begin{defn}[Mangasarian 1965]
        Hàm $f: \mathbb{R}^m \to \mathbb{R}$ được gọi là:
        \begin{itemize}
            \item<1-> \textbf{Lồi} (\textit{convex}) trên $C$ nếu $\forall \mathbf{x}, \mathbf{y} \in C, \lambda \in [0, 1]$:
            \[
                f(\lambda\mathbf{x} + (1 - \lambda)\mathbf{y}) \le \lambda f(\mathbf{x}) + (1 - \lambda)f(\mathbf{y}).
            \]
            \item<2-> \textbf{Giả lồi} (\textit{pseudoconvex}) trên $C$ nếu $\forall \mathbf{x}, \mathbf{y} \in C$:
            \[
                \nabla f(\mathbf{x})^T(\mathbf{y} - \mathbf{x}) \ge 0 \implies f(\mathbf{y}) \ge f(\mathbf{x}).
            \]
            \item<3-> \textbf{Tựa lồi} (\textit{quasiconvex}) trên $C$ nếu $\forall \mathbf{x}, \mathbf{y} \in C, \lambda \in [0, 1]$:
            \[
                f(\lambda\mathbf{x} + (1 - \lambda)\mathbf{y}) \le \max\{f(\mathbf{x}); f(\mathbf{y})\}.
            \]
        \end{itemize}
    \end{defn}
\end{frame}

\begin{frame}[t]{\textbf{\large 1.3.2 Đặc trưng và Tính chất Lipschitz}}
    \begin{block}{Mệnh đề 2: Đặc trưng tựa lồi}
        Hàm khả vi $f$ là tựa lồi trên $C$ khi và chỉ khi:
        \[
            f(\mathbf{y}) \le f(\mathbf{x}) \Rightarrow \nabla f(\mathbf{x})^T(\mathbf{y} - \mathbf{x}) \le 0, \quad \forall \mathbf{x}, \mathbf{y} \in C.
        \]
    \end{block}
    
    \pause
    
    \begin{alertblock}{Mối quan hệ bao hàm}
        Hàm lồi $\implies$ Hàm giả lồi $\implies$ Hàm tựa lồi
    \end{alertblock}

    \pause

    \begin{block}{Mệnh đề 3: Tính chất Lipschitz}
        Giả sử $\nabla f$ là liên tục $L$-Lipschitz trên $C$. Khi đó:
        \[
            |f(\mathbf{y}) - f(\mathbf{x}) - \nabla f(\mathbf{x})^T(\mathbf{y} - \mathbf{x})| \le \frac{L}{2} \|\mathbf{y} - \mathbf{x}\|^2.
        \]
    \end{block}
\end{frame}

\subsection{1.4 Bổ đề hỗ trợ chứng minh}

\begin{frame}[t]{\textbf{\large 1.4 Bổ đề hỗ trợ chứng minh}}
    \begin{lemma}[Xu 2002]
        Cho $\{a_k\}$ và $\{b_k\}$ là các dãy số dương sao cho:
        \[
            a_{k+1} \le a_k + b_k \quad \forall k \ge 0 \quad \text{và} \quad \sum_{k=0}^\infty b_k < \infty.
        \]
        Khi đó, tồn tại giới hạn $\lim_{k \to \infty} a_k = c \in \mathbb{R}$.
    \end{lemma}
\end{frame}
\section{2 Các kết quả chính}

\subsection{2.1 Thuật toán Gradient Descent Adaptive - GDA}
\begin{frame}[shrink]{\textbf{\large 2.1.1 Thuật toán Gradient Descent Adaptive - GDA}}
\footnotesize
\begin{algorithm}[H]
\caption{(Gradient Descent Adaptive Algorithm-GDA)}
\begin{algorithmic}[1]

\State \textbf{Input: $x^0 \in C$; $\lambda_0 > 0$; $\sigma, \kappa \in (0,1)$; $k=0$ }
\State \textbf{Output: Stationary point $x^*$}

\While{true}
    \State $x^{k+1} \gets P_C\big(x^k - \lambda_k \nabla f(x^k)\big)$

    \If{$f(x^{k+1}) \le f(x^k) - \sigma \langle \nabla f(x^k), x^k - x^{k+1} \rangle$}
        \State $\lambda_{k+1} \gets \lambda_k$
    \Else
        \State $\lambda_{k+1} \gets \kappa \lambda_k$
    \EndIf

    \If{$x^{k+1} = x^k$}
        \State \textbf{break}
    \EndIf

    \State $k \gets k + 1$
\EndWhile

\end{algorithmic}
\end{algorithm}
    
\end{frame}

\begin{frame}{\textbf{\large 2.1.2 Nhận xét}}
    \begin{exampleblock}{Nhận xét 1}
        Nếu thuật toán GDA dừng tại bước $k$, thì $\mathbf{x}^k$ là một điểm dừng của bài toán $OP(f,C)$.
    \end{exampleblock}
    \pause
    \begin{block}{Chứng minh:} Do $\mathbf{x}^{k+1} = P_C(\mathbf{x}^k - \lambda_k \nabla f(\mathbf{x}^k))$, áp dụng Mệnh đề 1-(ii), ta có:
    \begin{equation} \label{eq:1}
        \langle \mathbf{z} - \mathbf{x}^{k+1} ,  \mathbf{x}^k - \lambda_k \nabla f(\mathbf{x}^k) - \mathbf{x}^{k+1} \rangle  \le 0, \quad \forall \mathbf{z} \in C
    \end{equation}
    \pause
        Nếu $\mathbf{x}^{k+1} = \mathbf{x}^k$, ta thu được:
    \begin{equation} \label{eq:2}
        \langle \nabla f(\mathbf{x}^k), \mathbf{z} - \mathbf{x}^k \rangle \ge 0, \quad \forall \mathbf{z} \in C
    \end{equation}
    \pause
        Khi đó $\mathbf{x}^k$ là một điểm dừng của bài toán.
    \end{block}  %Hơn nữa, nếu $f$ là hàm giả lồi, từ (\ref{eq:2}) suy ra $f(\mathbf{z}) \ge f(\mathbf{x}^k)$ với mọi $\mathbf{z} \in C$, hay $\mathbf{x}^k$ là một nghiệm của $OP(f, C)$.
    
    %Giả sử thuật toán sinh ra một dãy vô hạn. Chúng ta sẽ chứng minh rằng dãy này hội tụ về một nghiệm của bài toán \hyperlink{target:OP}{OP($f, C$)}.
\end{frame}



\begin{frame}{\textbf{\large 2.1.3 Định lý về sự hội tụ của dãy}}
    \begin{thm}
        Giả sử dãy $\{\mathbf{x}^k\}$ được sinh bởi Thuật toán GDA. Khi đó, dãy $\{f(\mathbf{x}^k)\}$ hội tụ và mỗi điểm giới hạn (nếu có) của dãy $\{\mathbf{x}^k\}$ là một điểm dừng của bài toán.
        Hơn nữa:
        \begin{itemize}
            \item Nếu $f$ là hàm gần lồi trên $C$, thì dãy $\{\mathbf{x}^k\}$ hội tụ về một điểm dừng của bài toán.
            \item Nếu $f$ là hàm giả lồi trên $C$, thì dãy $\{\mathbf{x}^k\}$ hội tụ về một nghiệm của bài toán.
        \end{itemize}
    \end{thm}
\end{frame}

\begin{frame}{\textbf{\large 2.1.4 Chứng minh định lý 1}}
    \textbf{1. Bất đẳng thức cơ bản:}
    
    Kết hợp tính chất $L$-Lipschitz và tính chất của phép chiếu $P_C$:
    \begin{equation*}
        \begin{cases}
             f(\mathbf{x}^{k+1}) \le f(\mathbf{x}^k) + \langle \nabla f(\mathbf{x}^k), \mathbf{x}^{k+1} - \mathbf{x}^k \rangle + \frac{L}{2} \|\mathbf{x}^{k+1} - \mathbf{x}^k\|^2 \\
             \langle \nabla f(\mathbf{x}^k), \mathbf{x}^{k+1} - \mathbf{x}^k \rangle \le -\frac{1}{\lambda_k} \|\mathbf{x}^{k+1} - \mathbf{x}^k\|^2
        \end{cases}
    \end{equation*}

    \pause
    $\implies$ Ta thu được đánh giá quan trọng:
    \begin{equation} \label{eq:main_descent}
        \boxed{ f(\mathbf{x}^{k+1}) \le f(\mathbf{x}^k) - \sigma \langle \nabla f(\mathbf{x}^k),\mathbf{x}^k - \mathbf{x}^{k+1} \rangle - \left( \frac{1-\sigma}{\lambda_k} - \frac{L}{2} \right) \|\mathbf{x}^{k+1} - \mathbf{x}^k\|^2 }
    \end{equation}
\end{frame}

\begin{frame}{\textbf{\large 2.1.4 Chứng minh đinh lý 1}}
    \textbf{2. Sự hội tụ: }
    
    Giả sử $\lambda_k \to 0$, khi đó tồn tại $k_0 \in \mathbb{N}$ thỏa mãn
    \begin{equation} \label{eq:4}
        f(\mathbf{x}^{k+1}) \le f(\mathbf{x}^k) - \sigma \langle \nabla f(\mathbf{x}^k), \mathbf{x}^k-\mathbf{x}^{k+1} \rangle \quad \forall k \ge k_0. 
    \end{equation}
    
    Theo cách xây dựng $k_0$, bước nhảy không suy biến về 0:
    $$ \exists k_1: \lambda_k = \lambda_{k_1} = \bar{\lambda} > 0, \quad \forall k \ge k_1. $$ và thỏa mãn (\ref{eq:4})
    
    Từ $\langle \nabla f(\mathbf{x}^k),\mathbf{x}^k - \mathbf{x}^{k+1} \rangle \ge 0$ ta suy ra {$f(x^k)$} hội tụ và:
    $$ \sum_{k=0}^{\infty} \|\mathbf{x}^{k+1} - \mathbf{x}^k\|^2 < \infty $$
    \begin{equation} \label{eq:5}
        \implies \lim_{k \to \infty} \|\mathbf{x}^{k+1} - \mathbf{x}^k\| = 0.
    \end{equation}
\end{frame}

\begin{frame}{\textbf{\large 2.1.4 Chứng minh đinh lý 1}}
    \textbf{3. Điểm dừng: }
    
    Từ tính chất phép chiếu với mọi $\mathbf{z} \in C$:
    \begin{equation} \label{eq:6}
        \|\mathbf{x}^{k+1} - \mathbf{z}\|^2 \le \|\mathbf{x}^k - \mathbf{z}\|^2 - \|\mathbf{x}^{k+1} - \mathbf{x}^k\|^2 + 2\lambda_k \langle \nabla f(\mathbf{x}^k), \mathbf{z} - \mathbf{x}^{k+1} \rangle
    \end{equation}
    
    Xét dãy con $\mathbf{x}^{k_i} \to \bar{\mathbf{x}}$. Cho $i \to \infty$, sử dụng (\ref{eq:5}) và tính liên tục của $\nabla f$:
    \begin{equation}
         \langle \nabla f(\bar{\mathbf{x}}), \mathbf{z} - \bar{\mathbf{x}} \rangle \ge 0, \quad \forall \mathbf{z} \in C.
    \end{equation}
    $\implies \bar{\mathbf{x}}$ là một \textbf{điểm dừng} của bài toán.
\end{frame}

\begin{frame}{\textbf{\large 2.1.4 Chứng minh đinh lý 1}}
    \textbf{4. Trường hợp $f$ Gần lồi / Giả lồi: }
    
    Đặt $U := \{\mathbf{x} \in C : f(\mathbf{x}) \le f(\mathbf{x}^k), \forall k\}$. Với mọi $\hat{\mathbf{x}} \in U$:
    $$ \langle \nabla f(\mathbf{x}^k), \hat{\mathbf{x}} - \mathbf{x}^k \rangle \le 0 $$
    kết hợp (\ref{eq:6}) ta thu được:
    \begin{equation*}
        \|\mathbf{x}^{k+1} - \hat{\mathbf{x}}\|^2 \le \|\mathbf{x}^k - \hat{\mathbf{x}}\|^2 - \|\mathbf{x}^{k+1} - \mathbf{x}^k\|^2 + 2\lambda_k \langle \nabla f(\mathbf{x}^k), \mathbf{x}^k - \mathbf{x}^{k+1} \rangle
    \end{equation*}
    Áp dụng \textbf{Bổ đề 1} với $a_k = \|\mathbf{x}^{k+1} - \hat{\mathbf{x}}\|^2$ và $b_k = 2\lambda_k \nabla f(\mathbf{x}^k)^T (\mathbf{x}^k - \mathbf{x}^{k+1})$:
    $\implies \exists \lim_{k \to \infty} \|\mathbf{x}^k - \hat{\mathbf{x}}\| = 0$ .
    
    $\implies$ Dãy $\{\mathbf{x}^k\}$ hội tụ duy nhất về $\bar{\mathbf{x}}$.
    
   \textbf{Kết luận:} Dãy $\{\mathbf{x}^k\}$ hội tụ duy nhất về điểm dừng $\bar{\mathbf{x}}$. Nếu $f$ giả lồi, $\bar{\mathbf{x}}$ là \textbf{nghiệm tối ưu}. \qed
\end{frame}
\begin{frame}{\textbf{\large 2.1.5 Nhận xét}}
    \begin{exampleblock}{Nhận xét 2}
        \begin{itemize}
            \item Trong Thuật toán GDA, ta có thể chọn $\lambda_0 = \lambda$, với hằng số $\lambda \le 2(1 - \sigma)/L$. Thuật toán GDA áp dụng được cho bước nhảy hằng số $\lambda \le 2(1 - \sigma)/L$.
            \pause
            \item  Nếu giá trị hằng số Lipschitz $L$ được biết trước, ta có thể chọn bước nhảy hằng số $\lambda \in (0, 2/L)$ như trong thuật toán gradient descent (GD) để giải các bài toán quy hoạch lồi.
        \end{itemize}
    \end{exampleblock}
    
\end{frame}


\subsection{2.2 Thuật toán Gradient Descent - GD}
\begin{frame}[shrink]{\textbf{\large 2.2.1 Thuật toán Gradient Descent - GD}}
\footnotesize
\begin{algorithm}[H]
\caption{(Gradient Descent Algorithm-GD)}
\begin{algorithmic}[1]

\State \textbf{Input: $x^0 \in C$; $\lambda_0 \in (0, 2/L)$; $k=0$ }
\State \textbf{Output: Stationary point $x^*$}

\While{true}
    \State $x^{k+1} \gets P_C\big(x^k - \lambda_k \nabla f(x^k)\big)$
    \If{$x^{k+1} = x^k$}
        \State \textbf{break}
    \EndIf

    \State $k \gets k + 1$
\EndWhile

\end{algorithmic}
\end{algorithm}
\end{frame}

\begin{frame}{\textbf{\large 2.2.2 Hệ quả}}
    \begin{exampleblock}{Hệ quả 1: }
        Giả sử $f$ lồi, $C = \mathbb{R}^m$ và $\{x^k\}$ là dãy được sinh ra bởi thuật toán GDA. Khi đó,
        $$f(x^k) - f(x^*) = O\left(\frac{1}{k}\right).$$
        với $x^*$ là nghiệm tối ưu của bài toán.
    \end{exampleblock}
\end{frame}


\begin{frame}{\textbf{\large 2.2.3 Chứng minh hệ quả 1}}
    \textbf{1. Thiết lập bài toán:}
    Gọi $x^*$ là nghiệm tối ưu. Đặt $\Delta_k := f(x^k) - f(x^*) \ge 0$.

    \textbf{2. Các đánh giá cơ bản:}
    Từ (\ref{eq:4}) kết hợp $x^k-x^{k+1} = \lambda_k \nabla \ f(x^k)$, ta có:
    \begin{equation} \label{eq:rate_1}
        \Delta_{k+1} \le \Delta_k - \sigma \lambda \|\nabla f(x^k)\|^2, \quad \forall k \ge k_1
    \end{equation}

    Do dãy $\{ x^k\}$ bị chặn và $f$ lồi nên :
    \begin{equation} \label{eq:rate_2}
    \begin{split}
        \Delta_k &\le \langle \nabla f(x^k), x^k - x^* \rangle \\
                 &\le \underbrace{\|x^k - x^*\|}_{\le M} \|\nabla f(x^k)\| \implies \|\nabla f(x^k)\| \ge \frac{\Delta_k}{M}
    \end{split}
\end{equation}
trong đó $M := \sup \{ \|x^k - x^*\| : k \ge k_1 \} < \infty$.
\end{frame}

\begin{frame}{\textbf{\large 2.2.3 Chứng minh hệ quả 1}}
    \textbf{3. Bất đẳng thức truy hồi:}
    
    Thay (\ref{eq:rate_2}) vào (\ref{eq:rate_1}), đặt $Q := \frac{\sigma \lambda}{M^2}$:
    \begin{equation} \label{eq:rate_3}
        \Delta_{k+1} \le \Delta_k - \sigma \lambda \frac{\Delta_k^2}{M^2} = \Delta_k - Q \Delta_k^2
    \end{equation}

    \textbf{4. Đánh giá tốc độ:}
    Vì $\Delta_{k+1} \le \Delta_k$, chia hai vế (\ref{eq:rate_3}) cho $\Delta_k \Delta_{k+1}$:
    \begin{align*}
        \frac{1}{\Delta_{k+1}} \ge \frac{1}{\Delta_k} + Q \frac{\Delta_k}{\Delta_{k+1}} \ge \frac{1}{\Delta_k} + Q 
        &\implies \frac{1}{\Delta_k} \ge \frac{1}{\Delta_{k_1}} + (k - k_1)Q \\
        &\implies \Delta_k \le \frac{1}{(k - k_1)Q}
    \end{align*}

    \textbf{Kết luận:} $$ f(x^k) - f(x^*) = O\left(\frac{1}{k}\right).$$
\end{frame}
\subsection{2.3 Biến thể ngẫu nhiên của thuật toán GDA - Stochastic GDA}
\begin{frame}{\textbf{\large 2.3.1 Biến thể ngẫu nhiên SGDA}}
    \textbf{1. Động lực}
    Ứng dụng trong Học sâu quy mô lớn (Large-scale Deep Learning) nơi việc tính toán toàn bộ gradient $\nabla f(x)$ là bất khả thi.

    \textbf{2. Bài toán Tối ưu Ngẫu nhiên:}
    Thay vì tối thiểu hóa hàm mục tiêu xác định, ta xét bài toán kỳ vọng:
    \begin{equation} \label{eq:stochastic_prob}
         \min_{x} \mathbb{E}[f_{\xi}(x)].
    \end{equation}
    Trong đó:
    \begin{itemize}
        \item $x$: Biến quyết định (tham số mô hình).
        \item $\xi$: Biến ngẫu nhiên với phân phối xác định.
        \item $f_{\xi}(x)$: Hàm $L$-smooth đối với mỗi $\xi$.
    \end{itemize}
\end{frame}


\begin{frame}{\textbf{\large 2.3.1 Biến thể ngẫu nhiên SGDA}}
    \textbf{3. Xấp xỉ Gradient}
    Tại mỗi bước lặp $k$, thay vì tính $\nabla \Phi(x^k) = \mathbb{E}[\nabla f_{\xi}(x^k)]$, ta lấy mẫu:
    \begin{itemize}
        \item Lấy mẫu ngẫu nhiên $\xi$ tại mỗi lần lặp $k$. 
        \item Tính gradient xấp xỉ: $\nabla f_{\xi^k}(x^k)$.
    \end{itemize}

    \textbf{4. Các thành phần trong Học máy:}
    \begin{table}[]
        \centering
        \begin{tabular}{c|l}
            \textbf{Ký hiệu} & \textbf{Ý nghĩa thực tế} \\
            \hline
            $x$ & Trọng số mạng nơ-ron  \\
            $\xi$ & Tính ngẫu nhiên của việc chọn dữ liệu \\
            $f_{\xi}(x)$ & Hàm mất mát trên một batch dữ liệu \\
            $\Phi(x)$ & Hàm mất mát tổng quát 
        \end{tabular}
    \end{table}
    
    \textit{*Lưu ý: Các phân tích lý thuyết chặt chẽ cho SGDA được dành cho nghiên cứu tương lai.}
\end{frame}

\begin{frame}[shrink]{\textbf{\large 2.3.2 Thuật toán Stochastic Gradient Descent (SGDA)}}
\footnotesize
\begin{algorithm}[H]
\caption{(Stochastic Gradient Descent Algorithm-SGDA)}
\begin{algorithmic}[1]

\State \textbf{Input: $x^0 \in C$; $\lambda_0 > 0$; $\sigma, \kappa \in (0,1)$; $k=0$ }
\State \textbf{Output: Stationary point $x^*$}
\While{true}
    \State Lấy mẫu ngẫu nhiên $\xi_k$

    \State $   x_{k+1} \gets P_C\big(x_k - \lambda_k \nabla f_{\xi_k}(x_k)\big)  $
    \If{$f_{\xi_k}(x_{k+1}) \le f_{\xi_k}(x_k)
        - \sigma \langle \nabla f_{\xi_k}(x_k), x_k - x_{k+1} \rangle$}
        \State
        $    \lambda_{k+1} \gets \lambda_k $
    \Else
        \State
        $
            \lambda_{k+1} \gets \kappa \lambda_k
        $
    \EndIf
    \If{$x_{k+1} = x_k$}
        \State \textbf{break}
    \EndIf
    \State  $k \gets k + 1$
\EndWhile
\end{algorithmic}
\end{algorithm}
\end{frame}






\section{3 Các thí nghiệm số}
% =================================================================
% PHẦN 3.1: CÀI ĐẶT THỰC NGHIỆM & MỤC TIÊU 
% =================================================================
\begin{frame}{\textbf{\large  Cài đặt thực nghiệm \& Quy ước chung} }
    \vspace*{-0.9cm} 
    
    % [FIX] Đưa phần Mục tiêu so sánh từ slide dưới lên đây
    \begin{block}{\textbf{Mục tiêu}}
        \small
        Đánh giá hiệu quả của GDA thông qua đối sánh với các phương pháp:
        \begin{itemize}
            \item \textbf{Gradient Descent (GD)}: Cho \textbf{Bài toán Lồi} (Ví dụ 3).
            \item \textbf{Neurodynamic (RNN)}: Cho \textbf{Bài toán Phi/Giả lồi} (Ví dụ 1, 2, 4).
        \end{itemize}
    \end{block}

    \vspace{-0.2cm}
    \begin{block}{\textbf{Quy ước \& Tiêu chí dừng}} 
        \begin{itemize}
            \item \textbf{Điều kiện dừng thuật toán:}
            \begin{itemize}
                \item Số vòng lặp đạt giới hạn tối đa ($\text{Iterations} \ge \#\text{Iter}$).
                \item Hoặc khi \textbf{đạt điều kiện hội tụ} (Convergence condition met).
            \end{itemize}
            
            \vspace{0.1cm}
            \item \textbf{Các ký hiệu sử dụng:}
            \begin{itemize}
                \item $x^*$: Điểm giới hạn (nghiệm) của dãy lặp $\{x_k\}$.
                \item \textbf{Time}: Thời gian CPU thực chạy (không tính thời gian khởi tạo).
                \item \textbf{Value}: Giá trị hàm mục tiêu tại điểm dừng.
            \end{itemize}
        \end{itemize}
    \end{block}
\end{frame}

\iffalse
% =================================================================
% PHẦN 3.2: CHI TIẾT THUẬT TOÁN ĐỐI SÁNH (CHỈ CÒN RNN)
% =================================================================
\begin{frame}{\textbf{\large Mô hình đối sánh: Thuật toán Neurodynamic (RNN)} }
    \vspace*{-0.8cm} 
    
    % Slide này giờ rất thoáng, chỉ tập trung vào công thức toán
    \begin{block}{\textbf{Cơ sở lý thuyết RNN}}
        \small
        Xét bài toán với tập ràng buộc tổng quát: $C = \{x \in \mathbb{R}^n \mid g(x) \le 0, Ax = b\}$.
        \\
        Mô hình RNN (Liu et al., 2022) giải bài toán trên dưới dạng bao hàm vi phân:
        \vspace{-0.2cm} 
        \begin{equation*}
            \frac{d}{dt}x(t) \in -c(x(t))\nabla f(x(t)) - \partial \underbrace{P(x(t))}_{\text{Xử lý } g(x)} - \partial \underbrace{\|Ax(t) - b\|_1}_{\text{Xử lý } Ax=b}
        \end{equation*}
        
        \vspace{-0.1cm} 
        \begin{columns}[t]
            \column{0.48\textwidth}
            \centering
            \textbf{Hàm phạt} $P(x)$:
            \vspace{-0.2cm} 
            $$ \sum_{i=1}^{m} \max\{0, g_i(x)\} $$
            
            \column{0.48\textwidth}
            \centering
            \textbf{Hàm điều chỉnh} $\Psi(s)$:
            \vspace{-0.2cm} 
            $$ \begin{cases} 1, & s > 0 \\ [0,1], & s = 0 \\ 0, & s < 0 \end{cases} $$
        \end{columns}
    \end{block}
\end{frame}
\fi
% --- SLIDE 3.3: VÍ DỤ 1 - ĐỀ BÀI ---
\subsection{3.1.1 Ví dụ 1}
\begin{frame}{\textbf{\large 3.1 Ví dụ 1: Bài toán Phi lồi (Nonconvex)}}
    \vspace*{-0.8cm} % Kéo nội dung lên sát header
    
    \begin{block}{\textbf{Mô hình bài toán}}
        \small
        Hàm mục tiêu phi lồi:
        $$ \min f(x) = \frac{x_{1}^{2}+x_{2}^{2}+3}{1+2x_{1}+8x_{2}} $$
        
        Ràng buộc tập $C$:
        $$ C = \{(x_1, x_2)^\top \in \mathbb{R}^2 \mid -x_1^2 - 2x_1x_2 \le -4, \ x_1, x_2 \ge 0\} $$
    \end{block}

    \begin{alertblock}{\textbf{Nhận xét}}
        \small
        Hàm $f$ là giả lồi (pseudoconvex) trên $C$. Mục tiêu là so sánh khả năng hội tụ về nghiệm tối ưu toàn cục so với RNN.
    \end{alertblock}
\end{frame}


% --- SLIDE 3.1: VÍ DỤ 1 (ĐÃ THU GỌN & KÉO LÊN) ---
\begin{frame}{\textbf{\large 3.1.2 Ví dụ 1: Kết quả thực nghiệm}}
    \vspace*{-0.5cm} 
    \begin{columns}[T]
        % --- CỘT TRÁI ---
        \column{0.45\textwidth}
            \begin{block}{\textbf{Kết quả hội tụ}}
                \small
                \begin{itemize}
                    \item Nghiệm tối ưu $x^*$:
                    $$ (0.893870, 1.790527) $$
                    \item Value (GDA): \textbf{0.409362} (Tốt hơn RNN: 0.4101).
                \end{itemize}
            \end{block}
            
            % [FIX] Kéo khối này lên sát phía trên bằng vspace âm
            \vspace{-0.3cm} 
            
            \begin{alertblock}{\textbf{Đánh giá }}
                % [FIX] Dùng font nhỏ hơn (footnotesize) và ép dòng sát lại
                \footnotesize 
                \setlength{\itemsep}{1pt} 
                \begin{itemize}
                    \item \textbf{Tính ổn định:} Quỹ đạo nghiệm hội tụ mượt, không nhạy với nhiễu.
                    \item \textbf{Hiệu quả:} Cân bằng tốt giữa \textit{chất lượng nghiệm} và \textit{chi phí tính toán}.
                \end{itemize}
            \end{alertblock}

        % --- CỘT PHẢI ---
        \column{0.54\textwidth}
            \vspace{-0.2cm} 
            \begin{figure}
                \centering
                \includegraphics[width=\textwidth]{fig1.png}
                \caption{Quỹ đạo nghiệm hội tụ}
            \end{figure}
    \end{columns}
\end{frame}

% --- SLIDE 3.3: BẢNG SỐ LIỆU VÍ DỤ 1 (ĐÃ SỬA LỖI CHỮ TO) ---
\begin{frame}{\textbf{\large 3.1.3 Ví dụ 1: Thống kê chi tiết}}
    \vspace*{-0.6cm} 
    \textbf{Bảng 1: Kết quả các lần chạy (Results of Iterations)}
    
    \begin{table}[]
        \centering
        % [FIX] Dùng font nhỏ vừa phải, KHÔNG dùng resizebox để tránh bị phóng đại
        \footnotesize 
        \renewcommand{\arraystretch}{1.2} 
        \setlength{\tabcolsep}{5pt} % Tăng khoảng cách cột cho thoáng
        \begin{tabular}{|c|c|c|c|l|}
        \hline
        \textbf{\#} & \textbf{Time (s)} & \textbf{Iters} & \textbf{Value} & \multicolumn{1}{c|}{\textbf{$x^{*t}$}} \\ \hline
        1 & 1.489085 & \textbf{28} & 0.409364 & [0.894247, 1.789397] \\
        2 & 1.235719 & 32 & \textbf{0.409362} & [0.893870, 1.790527] \\
        3 & 1.082686 & 31 & 0.409363 & [0.893946, 1.790301] \\
        4 & \textbf{1.047740} & 31 & 0.409363 & [0.894092, 1.789862] \\
        5 & 1.440842 & 32 & 0.409364 & [0.894284, 1.789285] \\
        6 & 1.139069 & 30 & 0.409367 & [0.895131, 1.786744] \\
        7 & 1.126395 & 32 & 0.409363 & [0.894047, 1.789995] \\
        8 & 1.564762 & 33 & 0.409363 & [0.893926, 1.790359] \\
        9 & 2.008716 & 33 & 0.409363 & [0.893998, 1.790141] \\
        10 & 1.510566 & 32 & \textbf{0.409362} & [0.893911, 1.790403] \\ \hline
        \end{tabular}
    \end{table}
    
    \footnotesize \centering
    \textit{ Dữ liệu thực nghiệm 10 lần chạy đầu tiên}
\end{frame}


% --- SLIDE 3.6: VÍ DỤ 2 - ĐỀ BÀI ---
\subsection{3.2 Ví dụ 2}
\begin{frame}{\textbf{\large 3.2.1 Ví dụ 2: Bài toán Giả lồi không trơn}}
    \vspace*{-0.8cm} % Kéo nội dung lên
    \small 
    \textbf{Hàm mục tiêu (Không trơn):}
    \begin{equation*}
        \min f(x) = \frac{e^{|x_2 - 3|} - 30}{x_1^2 + x_3^2 + 2x_4^2 + 4}
    \end{equation*}
    
    \vspace{-0.2cm}
    \textbf{Hệ ràng buộc phức tạp:}
    \vspace{-0.3cm}
    \begin{align*}
        g_1(x) &= (x_1 + x_3)^3 + 2x_4^2 \le 10 \\
        g_2(x) &= (x_2 - 1)^2 \le 1, \quad h(x) = 2x_1 + 4x_2 + x_3 = -1
    \end{align*}

    \textbf{Xử lý đạo hàm tại điểm không trơn:}
    \vspace{-0.2cm}
    $$ \nabla |x_2 - 3| = \frac{x_2 - 3}{|x_2 - 3|} \quad (\text{với } x_2 \neq 3) $$
    
    \textbf{Kết quả mong đợi:} Tìm được nghiệm tốt hơn so với RNN (vốn hay bị kẹt ở cực tiểu địa phương đối với hàm không trơn).
\end{frame}

% --- SLIDE 3.4: VÍ DỤ 2 (ĐÃ SỬA LỖI ĐÁNH GIÁ + VECTOR 1 DÒNG) ---
\begin{frame}{\textbf{\large 3.2.2 Ví dụ 2: Kết quả thực nghiệm}}
    \vspace*{-0.5cm} 
    \begin{columns}[T]
        % --- CỘT TRÁI (48%) ---
        \column{0.48\textwidth}
            \begin{block}{\textbf{Kết quả hội tụ}}
                \footnotesize 
                \begin{itemize}
                    \item Nghiệm tối ưu $x^*$:
                    % [FIX] Ép vector dài nằm gọn trên 1 dòng
                    \resizebox{0.95\linewidth}{!}{
                        $ (-1.069016, 0.417647, -0.532558, 0.000022)^\top $
                    }
                    
                    \item Value (GDA): \textbf{-3.090767}.
                    \item Value (RNN): -3.0849.
                \end{itemize}
            \end{block}
            
            % [FIX] Kéo khối đánh giá lên sát phía trên
            \vspace{-0.3cm}
            
            % [CONTENT] Nội dung lấy từ Report Page 13
            \begin{alertblock}{\textbf{Đánh giá }}
                \scriptsize % Font nhỏ gọn để không bị tràn
                \setlength{\itemsep}{2pt} % Giãn dòng nhẹ cho dễ đọc
                \begin{itemize}
                    \item \textbf{Tính ổn định:} Hội tụ về cùng một nghiệm tối ưu bất kể điểm khởi tạo ban đầu khác nhau.
                    \item \textbf{Hiệu quả:} GDA vượt trội RNN cả về \textbf{độ chính xác}, \textbf{thời gian tính toán} và cấu trúc đơn giản hơn.
                \end{itemize}
            \end{alertblock}

        % --- CỘT PHẢI (50%) ---
        \column{0.50\textwidth}
            \vspace{-0.2cm}
            \begin{figure}
                \centering
                \includegraphics[width=\textwidth]{fig2.png}
                \caption{Sự hội tụ đồng nhất của 4 biến số}
            \end{figure}
    \end{columns}
\end{frame}
% --- SLIDE 3.2: BẢNG 2 (ĐÃ IN ĐẬM GIÁ TRỊ TỐI ƯU) ---
\begin{frame}{\textbf{\large 3.2.3 Ví dụ 2: Thống kê kết quả}}
    \vspace*{-0.8cm} % Kéo nội dung lên cao
    \centering
    \textbf{Bảng 2: Kết quả tối ưu (Optimization Results)}
    
    \vspace{0.2cm} % Khoảng cách giữa tiêu đề và bảng
    
    % [FIX] Bỏ môi trường 'table', dùng trực tiếp makebox để căn giữa bảng to
    \makebox[\textwidth][c]{
        \footnotesize % [FIX] Tăng cỡ chữ lên \footnotesize cho to bằng Bảng 1
        \renewcommand{\arraystretch}{1.25} % [FIX] Giãn dòng 1.25 cho thoáng
        \setlength{\tabcolsep}{3pt} % Khoảng cách cột vừa phải
        
        \begin{tabular}{|c|c|c|c|l|}
        \hline
        \textbf{\#} & \textbf{Time (s)} & \textbf{Iters} & \textbf{Value} & \multicolumn{1}{c|}{\textbf{$x^{*t}$}} \\ \hline
        1 & 3.125859 & 38 & \textbf{-3.090767} & [-1.069016, 0.417647, -0.532558, 0.000022] \\
        2 & 3.376504 & 42 & \textbf{-3.090767} & [-1.069795, 0.417966, -0.532273, 0.000000] \\
        3 & 3.359701 & 41 & \textbf{-3.090767} & [-1.069610, 0.417896, -0.532363, 0.000014] \\
        4 & 3.596762 & 44 & \textbf{-3.090767} & [-1.069956, 0.418058, -0.532320, 0.000005] \\
        5 & 3.124562 & 41 & \textbf{-3.090767} & [-1.069584, 0.417901, -0.532434, 0.000007] \\
        6 & 3.700269 & 47 & \textbf{-3.090767} & [-1.070102, 0.418157, -0.532424, 0.000004] \\
        7 & 3.406564 & 44 & \textbf{-3.090767} & [-1.069881, 0.418059, -0.532475, 0.000001] \\
        8 & \textbf{2.954161} & \textbf{37} & -3.090768 & [-1.067557, 0.417538, -0.535037, 0.000020] \\
        9 & 3.307867 & 47 & \textbf{-3.090767} & [-1.070151, 0.418162, -0.532345, 0.000000] \\
        10 & 3.687243 & 44 & \textbf{-3.090767} & [-1.069958, 0.418055, -0.532303, 0.000009] \\ \hline
        \end{tabular}
    }
    
    \vspace{0.2cm} % [FIX] Đẩy dòng ghi chú xuống dưới, không bị đè lên bảng
    \footnotesize 
    \textit{Giá trị hàm mục tiêu hội tụ ổn định}
\end{frame}
\subsection*{Ví dụ 3}

\begin{frame}{\textbf{\large 3.3.1 Ví dụ 3: Bài toán giả lồi trơn}}
Cho $e := (1,\ldots,n) \in \mathbb{R}^n$, với $\alpha>0$, $\beta>0$ thỏa mãn
\[
2\alpha > 3\beta^{3/2}\sqrt{n}.
\]

Xét bài toán tối ưu $\mathrm{OP}(f,C)$ (Ví dụ 4.5 trong Ferreira và Sosa, 2022) với
\[
f(x) := a^\top x + \alpha x^\top x
+ \frac{\beta}{\sqrt{1+\beta x^\top x}}\, e^\top x.
\]

\[
C = \left\{ x \in \mathbb{R}_{++}^n : x_1 x_2 \cdots x_n \ge 1 \right\}.
\]
\end{frame}

\begin{frame}{\textbf{\large 3.3.2 Thiết lập tham số và thuật toán}}
\[
\beta = 0.741271, \qquad
\alpha = 3\beta^{3/2}\sqrt{n+1}.
\]

\[
L = 4\beta^{3/2}\sqrt{n} + 3\alpha.
\]

\textbf{So sánh hai phương pháp:}
\begin{itemize}
\item \textbf{GD:} $\lambda = 1/L$.
\item \textbf{GDA:} $\lambda_0 = 5/L$, điều chỉnh thích nghi.
\end{itemize}
\end{frame}

\begin{frame}{\textbf{\large 3.3.3 Kết quả thực nghiệm}}
\scriptsize
\begin{table}
\centering
\begin{tabular}{|r|rrr||rrr|}
\hline
$n$ & \multicolumn{3}{c||}{GDA} & \multicolumn{3}{c|}{GD} \\ \cline{2-7}
 & $f(x^*)$ & \#Iter & Time(s)
 & $f(x^*)$ & \#Iter & Time(s) \\
\hline
10   & 80.080578 & 6 & 0.5927 & 80.080615 & 6 & 0.2399 \\
20   & 219.714476 & 6 & 2.2912 & 219.714478 & 7 & 0.9540 \\
50   & 851.700044 & 7 & 3.3230 & 851.700047 & 8 & 2.6453 \\
100  & 2390.314066 & 7 & 11.9520 & 2390.314086 & 8 & 12.3069 \\
200  & 6721.095888 & 8 & 71.6870 & 6721.095916 & 9 & 64.0119 \\
600  & 34711.054243 & 8 & 932.1330 & 34711.054356 & 10 & 529.1590 \\
\hline
\end{tabular}
\caption{So sánh GDA và GD}
\end{table}
\end{frame}

\begin{frame}{\textbf{\large 3.3.4 Nhận xét}}
\begin{itemize}
    \item GDA và GD cho giá trị hàm mục tiêu tối ưu gần như tương đương.
    \item GDA thường hội tụ với ít vòng lặp hơn so với GD.
    \item Hiệu quả của GDA thể hiện rõ hơn khi kích thước bài toán tăng.
    \item Cơ chế bước học thích nghi giúp GDA cải thiện hiệu suất tính toán.
\end{itemize}
\end{frame}




% ==========================
\subsection*{Ví dụ 4}


\begin{frame}{\textbf{\large 3.4.1 Ví dụ 4: Bài toán tối ưu giả lồi trơn với các ràng buộc lồi}}
Xét bài toán $\mathrm{OP}(f,C)$ với hàm mục tiêu
\[
f(x) = -\exp\!\left(-\sum_{i=1}^n \frac{x_i^2}{e_i^2}\right),
\]
trong đó $x\in\mathbb{R}^n$, $e=(e_1,\ldots,e_n)^\top$, $e_i>0$.

Hàm $f$ là giả lồi trên miền ràng buộc lồi
\[
C := \{ Ax=b,\; g(x)\le0 \}.
\]
\end{frame}

% ==========================
\begin{frame}{\textbf{\large 3.4.2 Cấu trúc ràng buộc}}
Ma trận $A=(a_1,\ldots,a_n)\in\mathbb{R}^{1\times n}$ được xác định bởi
\[
a_i=
\begin{cases}
1, & 1\le i\le n/2,\\
3, & n/2 < i \le n,
\end{cases}
\qquad
b=16.
\]

Các ràng buộc bất đẳng thức:
\[
g_i(x)=\sum_{j=1}^{10} x_{10(i-1)+j}^2 - 20,
\qquad i=1,2,\ldots,\frac{n}{10}.
\]
\end{frame}

% ==========================
\begin{frame}{\textbf{\large 3.4.3 Thiết lập so sánh thuật toán}}
So sánh hai phương pháp:
\begin{itemize}
\item \textbf{GDA}: thuật toán đề xuất.
\item \textbf{RNN}: thuật toán của Liu et al.\ (2022).
\end{itemize}

Chỉ số đánh giá được sử dụng:
\[
-\ln(-f(x^*)),
\]
trong đó $x^*$ là nghiệm gần đúng thu được từ mỗi thuật toán.

Hàm $-\ln(-z)$ là đơn điệu tăng với mọi $z<0$.
\end{frame}

% ==========================
\begin{frame}{\textbf{\large 3.4.4 Kết quả thực nghiệm}}
\scriptsize
\begin{table}
\centering
\begin{tabular}{|c|c|c|c|c|c|c|}
\hline
$n$ & \multicolumn{3}{c|}{GDA} & \multicolumn{3}{c|}{RNN} \\ \cline{2-7}
 & $-\ln(-f(x^*))$ & \#Iter & Time
 & $-\ln(-f(x^*))$ & \#Iter & Time \\
\hline
10  & 5.1200 & 50 & 0.54 & 5.2014 & 10000 & 1123 \\
20  & 2.5600 & 50 & 231.4 & 2.9004 & 10000 & 1264 \\
30  & 1.7066 & 50 & 312 & 5.2014 & 10000 & 1123 \\
40  & 1.2800 & 50 & 309.2 & 6.5283 & 10000 & 1634 \\
50  & 1.0240 & 50 & 261.6 & 1.4141 & 10000 & 1367 \\
60  & 0.8533 & 50 & 348.5 & 1.4141 & 10000 & 1367 \\
70  & 0.7314 & 500 & 78.4 & 11.1273 & 10000 & 1443.2 \\
\hline
\end{tabular}
\caption{Kết quả tính toán cho Ví dụ 4}
\end{table}
\end{frame}

% ==========================
\begin{frame}{\textbf{\large 3.4.5 Nhận xét}}
\begin{itemize}
\item GDA cho \textbf{giá trị tối ưu gần đúng tốt hơn} so với RNN.
\item GDA hội tụ với \textbf{số vòng lặp ít hơn đáng kể}.
\item \textbf{Thời gian tính toán thấp hơn nhiều}, đặc biệt khi $n$ lớn.
\end{itemize}

Kết quả cho thấy GDA vượt trội hơn RNN trong các bài toán giả lồi
kích thước lớn.
\end{frame}

\section{4 Applications to machine learning}
\begin{frame}[t]{\textbf{\large 4. Ứng dụng trong Học máy (Machine Learning)}}
	\begin{block}{Mục tiêu nghiên cứu}
		Phương pháp được đề xuất, giống như thuật toán GD, có nhiều ứng dụng trong học máy. Chúng tôi phân tích ba ứng dụng phổ biến để chứng minh độ chính xác và hiệu quả tính toán so với các phương pháp thay thế khác:
	\end{block}
	
	\vspace{0.5cm}
	\begin{enumerate}
		\item \textbf{Lựa chọn đặc trưng có giám sát} (Supervised feature selection).
		\item \textbf{Hồi quy Logistic} (Regression).
		\item \textbf{Phân loại} (Classification).
	\end{enumerate}
\end{frame}

\subsection{4.1 Cơ sở lý thuyết và phương pháp tiếp cận}
\begin{frame}[t]{\textbf{\large Cơ sở lý thuyết và Phương pháp tiếp cận}}
	\begin{itemize}
		\item \textbf{1. Lựa chọn đặc trưng:} 
		\begin{itemize}
			\item \textit{Mô hình:} Cực tiểu hóa hàm phân thức \textbf{giả lồi} trên tập lồi (lớp con của bài toán $OP(f, C)$).
			\item \textit{Mục đích:} So sánh với phương pháp thần kinh động lực học (neurodynamic).
		\end{itemize}
		\vspace{0.3cm}
		
		\item \textbf{2. Hồi quy Logistic đa biến:}
		\begin{itemize}
			\item \textit{Mô hình:} Bài toán \textbf{quy hoạch lồi}.
			\item \textit{Giải pháp:} Dùng GDA và các biến thể GD.
		\end{itemize}
		\vspace{0.3cm}
		
		\item \textbf{3. Mạng Nơ-ron (Phân loại ảnh):}
		\begin{itemize}
			\item \textit{Mô hình:} Hàm mục tiêu \textbf{không lồi, không tựa lồi}.
			\item \textit{Giải pháp:} Dùng biến thể ngẫu nhiên \textbf{SGDA} (heuristic).
			\item \textit{Hội tụ:} Nếu dãy điểm có giới hạn $\rightarrow$ hội tụ về điểm dừng (Định lý 1).
		\end{itemize}
	\end{itemize}
\end{frame}

\subsection{4.2 Lựa chọn đặc trưng có giám sát}
\begin{frame}[t]{\textbf{\large 4.2.1 Lựa chọn đặc trưng có giám sát: Thiết lập}}
	\textbf{Đầu vào:}
	\begin{itemize}
		\item Tập $p$-đặc trưng $\mathcal{F} = \{F_1, ..., F_p\}$.
		\item Tập $n$-mẫu $\{(x_i, y_i) | i = 1, ..., n\}$.
		\item $x_i$: vector đặc trưng $p$-chiều; $y_i \in \{1, ..., m\}$: nhãn lớp.
	\end{itemize}
	
	\vspace{0.5cm}
	\textbf{Mục tiêu:} Chọn tập con tối ưu $\{F_1, ..., F_k\} \subseteq \mathcal{F}$ thỏa mãn:
	\begin{enumerate}
		\item \textbf{Sự dư thừa ít nhất (Min Redundancy):} Cực tiểu hóa $w^T Q w$ ($Q$ là ma trận bán xác định dương).
		\item \textbf{Mức độ liên quan cao nhất (Max Relevance):} Cực đại hóa $\rho^T w$ ($\rho$ là vector tham số liên quan).
	\end{enumerate}
\end{frame}

\begin{frame}[t]{\textbf{\large 4.2.2 Mô hình bài toán tương đương}}
	Kết hợp hai mục tiêu trên, ta có bài toán tối ưu hàm phân thức:
	
	\begin{equation}
		\label{eq:14}
		\begin{aligned}
			& \text{minimize} \quad \frac{w^T Q w}{\rho^T w} \\
			& \text{subject to} \quad e^T w = 1, \quad w \geq 0
		\end{aligned}
	\end{equation}
	
	Trong đó:
	\begin{itemize}
		\item $w = (w_1, ..., w_p)^T$: Vector điểm số đặc trưng cần xác định.
	\end{itemize}
	
	\vspace{0.5cm}
	\textbf{Nhận xét quan trọng:}
	Vì hàm mục tiêu là phân thức của một hàm lồi trên một hàm tuyến tính dương $\rightarrow$ Nó là \textbf{hàm giả lồi} trên tập ràng buộc.
	$\Rightarrow$ Có thể giải bằng \textbf{Thuật toán GDA}.
\end{frame}

\begin{frame}[t]{\textbf{\large 4.2.3 Xây dựng tham số thực nghiệm (Dataset: Parkinsons)}}
	\textbf{1. Ma trận hệ số tương đồng $Q = \delta I_p + S$:}
	Với $S = (s_{ij})$ xác định bởi:
	$$ s_{ij} = \max \left\{ 0, \frac{I(F_i; F_j; y)}{H(F_i) + H(F_j)} \right\} $$
	
	\footnotesize
	Trong đó:
	\begin{itemize}
		\item Entropy thông tin: $H(\hat{X}) = - \sum_{\hat{x} \in \hat{X}} p(\hat{x}) \log p(\hat{x})$
		\item Đa thông tin: $I(\hat{X}; \hat{Y}; \hat{Z}) = I(\hat{X}; \hat{Y}) - I(\hat{X}; \hat{Y} | \hat{Z})$
		\item Thông tin tương hỗ có điều kiện:
		$ I(\hat{X}; \hat{Y} | \hat{Z}) = \sum \sum \sum p(\hat{x}, \hat{y}, \hat{z}) \log \frac{p(\hat{x}, \hat{y} | \hat{z})}{p(\hat{x} | \hat{z})p(\hat{y} | \hat{z})} $
	\end{itemize}
	\normalsize
	
	\vspace{0.2cm}
	\textbf{2. Vector mức độ liên quan $\rho$ (Fisher score):}
	$$ \rho(F_i) = \frac{\sum_{j=1}^{K} n_j (\mu_{ij} - \mu_i)^2}{\sum_{j=1}^{K} n_j \sigma_{ij}^2} $$
\end{frame}


\begin{frame}[t]{\textbf{\large 4.2.4 Kết quả so sánh thực nghiệm}}
	So sánh Thuật toán GDA (đề xuất) và Thuật toán RNN (Wang et al. 2021):
	
	\begin{table}[]
		\centering
		\begin{tabular}{|l|c|c|}
			\hline
			\textbf{Chỉ số} & \textbf{Thuật toán GDA} & \textbf{Thuật toán RNN} \\ \hline
			Giá trị tối ưu $f(w^*)$ & \textbf{0.153478} & 0.153911 \\ \hline
			Thời gian tính toán ($T$) & \textbf{10.034473s} & 49.324688s  \\ \hline
		\end{tabular}
	\end{table}
	
	\vspace{0.5cm}
	\textbf{Kết luận:}
	\begin{itemize}
		\item Thuật toán GDA tìm được giá trị tối ưu tốt hơn (nhỏ hơn).
		\item Thời gian tính toán nhanh hơn gần gấp đôi so với RNN.
		\item GDA vượt trội cả về độ chính xác và hiệu quả tính toán.
	\end{itemize}
\end{frame}
\begin{frame}[t]{\textbf{\large 4.2.4 Kết quả so sánh thực nghiệm}}
    \begin{center}
        \includegraphics[width=\linewidth]{res/Figure_5_1.png}
        {Hình 3: Kết quả cho bài toán lựa chọn đặc trưng}
    \end{center}
\end{frame}
\subsection{4.3 Hồi quy Logistic}
\begin{frame}[t]{\textbf{\large 4.3.1 Hồi quy Logistic đa biến: Thiết lập}}
	\textbf{Bài toán:}
	\begin{itemize}
		\item Dữ liệu: $N$ quan sát $(a_i, b_i) \in \mathbb{R}^d \times \mathbb{R}$.
		\item Hàm mất mát (Cross-entropy + chuẩn hóa $L_2$):
	\end{itemize}
	\begin{align*}
		\bar{J}(x) = -\sum_{i=1}^{N} \left( b_i \log(\sigma(-x^T a_i)) + (1-b_i)\log(1-\sigma(-x^T a_i)) \right) + \frac{1}{2N}\|x\|^2
	\end{align*}
	
	\textbf{Tham số thuật toán:}
	\begin{itemize}
		\item Hệ số Lipschitz ước lượng: $L \approx \frac{1}{2N}(\|A\|^2/2 + 1)$.
		\item So sánh với:
		\begin{enumerate}
			\item GD (bước nhảy $1/L$).
			\item Nesterov's accelerated method.
		\end{enumerate}
	\end{itemize}
\end{frame}
\begin{frame}[t]{\textbf{\large 4.3.2 Kết quả Hồi quy Logistic}}
    \begin{columns}[T]
        \begin{column}{0.48\textwidth}
            \centering
            \includegraphics[width=\linewidth, height=4cm, keepaspectratio]{res/Figure_5_2_mush_compare.png}
        \end{column}
        \begin{column}{0.48\textwidth}
            \centering
            \includegraphics[width=\linewidth, height=4cm, keepaspectratio]{res/Figure_5_2_w8a_compare.png}
        \end{column}
    \end{columns}
    {\centering \footnotesize Hình 4: Kết quả so sánh đối với dataset Mushrooms (bên trái) và W8a (bên phải) \par}

    \vspace{0.1cm}
    \textbf{Phân tích biểu đồ (Dataset: Mushrooms và W8a):}
    \small 
    \begin{itemize}
        \setlength\itemsep{0em}
        \item GDA vượt trội hơn GD và Nesterov về giá trị hàm mục tiêu.
        \item \textbf{Hình 5} minh họa cơ chế tự thích nghi: Bước nhảy giảm dần từ giá trị ban đầu (với hệ số $\kappa = 0.75$).
        \item \textbf{Hình 6} trình bày sự giảm kích thước bước nhảy từ một kích thước bước nhảy ban đầu liên quan đến các kết quả trong Hình 5.
    \end{itemize}
\end{frame}

\begin{frame}{\textbf{\large 4.3.3 Kết quả Hồi quy Logistic}}
    \begin{columns}[T]
        \begin{column}{0.5\textwidth}
            \centering
            % Tăng chiều cao ảnh lên để tận dụng không gian slide
            \includegraphics[width=\linewidth, height=6cm, keepaspectratio]{res/Figure_5_2_mush_compare_all.png} 
        \end{column}
        \begin{column}{0.5\textwidth}
            \centering
            \includegraphics[width=\linewidth, height=6cm, keepaspectratio]{res/Figure_5_2_w8a_compare_all.png}
        \end{column}
    \end{columns}
    
    \vspace{0.5cm} % Tạo khoảng cách giữa ảnh và chú thích
    
    \begin{center}
        \footnotesize {Hình 5: Kết quả so sánh đối với dataset Mushrooms (bên trái) và W8a (bên phải)}
    \end{center}
\end{frame}
\begin{frame}{\textbf{\large 4.3.3 Kết quả Hồi quy Logistic (tiếp theo)}}
    \begin{columns}[T]
        \begin{column}{0.5\textwidth}
            \centering
            \includegraphics[width=\linewidth, height=6cm, keepaspectratio]{res/Figure_5_2_mush_lr.png}
        \end{column}
        \begin{column}{0.5\textwidth}
            \centering
            \includegraphics[width=\linewidth, height=6cm, keepaspectratio]{res/Figure_5_2_w8a_lr.png}
        \end{column}
    \end{columns}
    
    \vspace{0.5cm} 
    
    \begin{center}
        \footnotesize {Hình 6: Sự thay đổi bước nhảy đối với dataset Mushrooms (bên trái) và W8a (bên phải)}
    \end{center}
\end{frame}

\subsection{4.4 Phân loại}
\begin{frame}[t]{\textbf{\large 4.4.1 Mạng Nơ-ron cho phân loại ảnh}}
	\textbf{Mục tiêu:}
	Triển khai thuật toán đề xuất vào mô hình huấn luyện mạng nơ-ron (bài toán không lồi).
	
	\vspace{0.3cm}
	\textbf{Cấu hình thực nghiệm:}
	\begin{itemize}
		\item \textbf{Mô hình:} Kiến trúc ResNet-18 tiêu chuẩn (PyTorch).
		\item \textbf{Dữ liệu:} Cifar10 (Phân loại ảnh).
		\item \textbf{Hàm mất mát:} Cross-entropy.
		\item \textbf{Tham số:} Sử dụng thiết lập mặc định của Adam.
	\end{itemize}
	
	\vspace{0.3cm}
	\textbf{Phương pháp so sánh:}
	\begin{itemize}
		\item Biến thể ngẫu nhiên của GDA (\textbf{Thuật toán SGDA}).
		\item So với: Stochastic Gradient Descent (\textbf{SGD}).
	\end{itemize}
\end{frame}


\begin{frame}[t]{\textbf{\large 4.4.2 Kết quả huấn luyện Mạng Nơ-ron}}
	\textbf{Kết quả (Hình 7 \& 8):}
	SGDA vượt trội hơn SGD về cả hai chỉ số quan trọng:
	
	\begin{columns}
		\column{0.5\textwidth}
		\begin{figure}
			\centering
			\includegraphics[width=\textwidth]{res/ann_comparison_test_accuracy.png}
		\end{figure}
		\begin{center}
            \footnotesize{ Hình 7: Độ chính xác kiểm thử (test accuracy) theo số vòng lặp khi huấn luyện ResNet-18.}
		\end{center}
		\column{0.5\textwidth}
		\begin{figure}
			\centering
			\includegraphics[width=\textwidth]{res/ann_comparison_training_loss.png}	
            \begin{center}
                \footnotesize{Hình 8: Hàm mất mất huấn luyện (training loss) theo số vòng lặp cho ResNet-18.}
            \end{center}
		\end{figure}
	\end{columns}
\end{frame}




\section{5 Tài liệu tham khảo}
\begin{frame}[t]{\textbf{\large 5 Tài liệu tham khảo}}
    \begin{itemize}
        \item \url {https://github.com/nguyenphongg233/NMPPTU-Team3-SAAQP}
        \item \url {https://www.overleaf.com/project/6925bea64fbe6a88cd865b51}
    \end{itemize}
\end{frame}

\begin{frame}[c, plain] 
    \centering 
    
    \Huge \color{hustred}{\textbf{Cảm ơn mọi người đã chú ý lắng nghe!}}
    
\end{frame}


\end{document}
