
\subsection{2.1 Thuật toán Gradient Descent Adaptive - GDA}
\begin{frame}[shrink]{\textbf{\large 2.1.1 Thuật toán Gradient Descent Adaptive - GDA}}
\footnotesize
\begin{algorithm}[H]
\caption{(Gradient Descent Adaptive Algorithm-GDA)}
\begin{algorithmic}[1]

\State \textbf{Input: $x^0 \in C$; $\lambda_0 > 0$; $\sigma, \kappa \in (0,1)$; $k=0$ }
\State \textbf{Output: Stationary point $x^*$}

\While{true}
    \State $x^{k+1} \gets P_C\big(x^k - \lambda_k \nabla f(x^k)\big)$

    \If{$f(x^{k+1}) \le f(x^k) - \sigma \langle \nabla f(x^k), x^k - x^{k+1} \rangle$}
        \State $\lambda_{k+1} \gets \lambda_k$
    \Else
        \State $\lambda_{k+1} \gets \kappa \lambda_k$
    \EndIf

    \If{$x^{k+1} = x^k$}
        \State \textbf{break}
    \EndIf

    \State $k \gets k + 1$
\EndWhile

\end{algorithmic}
\end{algorithm}
    
\end{frame}

\begin{frame}{\textbf{\large 2.1.2 Nhận xét}}
    \begin{exampleblock}{Nhận xét 1}
        Nếu thuật toán GDA dừng tại bước $k$, thì $\mathbf{x}^k$ là một điểm dừng của bài toán $OP(f,C)$.
    \end{exampleblock}
    \pause
    \begin{block}{Chứng minh:} Do $\mathbf{x}^{k+1} = P_C(\mathbf{x}^k - \lambda_k \nabla f(\mathbf{x}^k))$, áp dụng Mệnh đề 1-(ii), ta có:
    \begin{equation} \label{eq:1}
        \langle \mathbf{z} - \mathbf{x}^{k+1} ,  \mathbf{x}^k - \lambda_k \nabla f(\mathbf{x}^k) - \mathbf{x}^{k+1} \rangle  \le 0, \quad \forall \mathbf{z} \in C
    \end{equation}
    \pause
        Nếu $\mathbf{x}^{k+1} = \mathbf{x}^k$, ta thu được:
    \begin{equation} \label{eq:2}
        \langle \nabla f(\mathbf{x}^k), \mathbf{z} - \mathbf{x}^k \rangle \ge 0, \quad \forall \mathbf{z} \in C
    \end{equation}
    \pause
        Khi đó $\mathbf{x}^k$ là một điểm dừng của bài toán.
    \end{block}  %Hơn nữa, nếu $f$ là hàm giả lồi, từ (\ref{eq:2}) suy ra $f(\mathbf{z}) \ge f(\mathbf{x}^k)$ với mọi $\mathbf{z} \in C$, hay $\mathbf{x}^k$ là một nghiệm của $OP(f, C)$.
    
    %Giả sử thuật toán sinh ra một dãy vô hạn. Chúng ta sẽ chứng minh rằng dãy này hội tụ về một nghiệm của bài toán \hyperlink{target:OP}{OP($f, C$)}.
\end{frame}



\begin{frame}{\textbf{\large 2.1.3 Định lý về sự hội tụ của dãy}}
    \begin{thm}
        Giả sử dãy $\{\mathbf{x}^k\}$ được sinh bởi Thuật toán GDA. Khi đó, dãy $\{f(\mathbf{x}^k)\}$ hội tụ và mỗi điểm giới hạn (nếu có) của dãy $\{\mathbf{x}^k\}$ là một điểm dừng của bài toán.
        Hơn nữa:
        \begin{itemize}
            \item Nếu $f$ là hàm gần lồi trên $C$, thì dãy $\{\mathbf{x}^k\}$ hội tụ về một điểm dừng của bài toán.
            \item Nếu $f$ là hàm giả lồi trên $C$, thì dãy $\{\mathbf{x}^k\}$ hội tụ về một nghiệm của bài toán.
        \end{itemize}
    \end{thm}
\end{frame}

\begin{frame}{\textbf{\large 2.1.4 Chứng minh định lý 1}}
    \textbf{1. Bất đẳng thức cơ bản:}
    
    Kết hợp tính chất $L$-Lipschitz và tính chất của phép chiếu $P_C$:
    \begin{equation*}
        \begin{cases}
             f(\mathbf{x}^{k+1}) \le f(\mathbf{x}^k) + \langle \nabla f(\mathbf{x}^k), \mathbf{x}^{k+1} - \mathbf{x}^k \rangle + \frac{L}{2} \|\mathbf{x}^{k+1} - \mathbf{x}^k\|^2 \\
             \langle \nabla f(\mathbf{x}^k), \mathbf{x}^{k+1} - \mathbf{x}^k \rangle \le -\frac{1}{\lambda_k} \|\mathbf{x}^{k+1} - \mathbf{x}^k\|^2
        \end{cases}
    \end{equation*}

    \pause
    $\implies$ Ta thu được đánh giá quan trọng:
    \begin{equation} \label{eq:main_descent}
        \boxed{ f(\mathbf{x}^{k+1}) \le f(\mathbf{x}^k) - \sigma \langle \nabla f(\mathbf{x}^k),\mathbf{x}^k - \mathbf{x}^{k+1} \rangle - \left( \frac{1-\sigma}{\lambda_k} - \frac{L}{2} \right) \|\mathbf{x}^{k+1} - \mathbf{x}^k\|^2 }
    \end{equation}
\end{frame}

\begin{frame}{\textbf{\large 2.1.4 Chứng minh đinh lý 1}}
    \textbf{2. Sự hội tụ: }
    
    Giả sử $\lambda_k \to 0$, khi đó tồn tại $k_0 \in \mathbb{N}$ thỏa mãn
    \begin{equation} \label{eq:4}
        f(\mathbf{x}^{k+1}) \le f(\mathbf{x}^k) - \sigma \langle \nabla f(\mathbf{x}^k), \mathbf{x}^k-\mathbf{x}^{k+1} \rangle \quad \forall k \ge k_0. 
    \end{equation}
    
    Theo cách xây dựng $k_0$, bước nhảy không suy biến về 0:
    $$ \exists k_1: \lambda_k = \lambda_{k_1} = \bar{\lambda} > 0, \quad \forall k \ge k_1. $$ và thỏa mãn (\ref{eq:4})
    
    Từ $\langle \nabla f(\mathbf{x}^k),\mathbf{x}^k - \mathbf{x}^{k+1} \rangle \ge 0$ ta suy ra {$f(x^k)$} hội tụ và:
    $$ \sum_{k=0}^{\infty} \|\mathbf{x}^{k+1} - \mathbf{x}^k\|^2 < \infty $$
    \begin{equation} \label{eq:5}
        \implies \lim_{k \to \infty} \|\mathbf{x}^{k+1} - \mathbf{x}^k\| = 0.
    \end{equation}
\end{frame}

\begin{frame}{\textbf{\large 2.1.4 Chứng minh đinh lý 1}}
    \textbf{3. Điểm dừng: }
    
    Từ tính chất phép chiếu với mọi $\mathbf{z} \in C$:
    \begin{equation} \label{eq:6}
        \|\mathbf{x}^{k+1} - \mathbf{z}\|^2 \le \|\mathbf{x}^k - \mathbf{z}\|^2 - \|\mathbf{x}^{k+1} - \mathbf{x}^k\|^2 + 2\lambda_k \langle \nabla f(\mathbf{x}^k), \mathbf{z} - \mathbf{x}^{k+1} \rangle
    \end{equation}
    
    Xét dãy con $\mathbf{x}^{k_i} \to \bar{\mathbf{x}}$. Cho $i \to \infty$, sử dụng (\ref{eq:5}) và tính liên tục của $\nabla f$:
    \begin{equation}
         \langle \nabla f(\bar{\mathbf{x}}), \mathbf{z} - \bar{\mathbf{x}} \rangle \ge 0, \quad \forall \mathbf{z} \in C.
    \end{equation}
    $\implies \bar{\mathbf{x}}$ là một \textbf{điểm dừng} của bài toán.
\end{frame}

\begin{frame}{\textbf{\large 2.1.4 Chứng minh đinh lý 1}}
    \textbf{4. Trường hợp $f$ Gần lồi / Giả lồi: }
    
    Đặt $U := \{\mathbf{x} \in C : f(\mathbf{x}) \le f(\mathbf{x}^k), \forall k\}$. Với mọi $\hat{\mathbf{x}} \in U$:
    $$ \langle \nabla f(\mathbf{x}^k), \hat{\mathbf{x}} - \mathbf{x}^k \rangle \le 0 $$
    kết hợp (\ref{eq:6}) ta thu được:
    \begin{equation*}
        \|\mathbf{x}^{k+1} - \hat{\mathbf{x}}\|^2 \le \|\mathbf{x}^k - \hat{\mathbf{x}}\|^2 - \|\mathbf{x}^{k+1} - \mathbf{x}^k\|^2 + 2\lambda_k \langle \nabla f(\mathbf{x}^k), \mathbf{x}^k - \mathbf{x}^{k+1} \rangle
    \end{equation*}
    Áp dụng \textbf{Bổ đề 1} với $a_k = \|\mathbf{x}^{k+1} - \hat{\mathbf{x}}\|^2$ và $b_k = 2\lambda_k \nabla f(\mathbf{x}^k)^T (\mathbf{x}^k - \mathbf{x}^{k+1})$:
    $\implies \exists \lim_{k \to \infty} \|\mathbf{x}^k - \hat{\mathbf{x}}\| = 0$ .
    
    $\implies$ Dãy $\{\mathbf{x}^k\}$ hội tụ duy nhất về $\bar{\mathbf{x}}$.
    
   \textbf{Kết luận:} Dãy $\{\mathbf{x}^k\}$ hội tụ duy nhất về điểm dừng $\bar{\mathbf{x}}$. Nếu $f$ giả lồi, $\bar{\mathbf{x}}$ là \textbf{nghiệm tối ưu}. \qed
\end{frame}
\begin{frame}{\textbf{\large 2.1.5 Nhận xét}}
    \begin{exampleblock}{Nhận xét 2}
        \begin{itemize}
            \item Trong Thuật toán GDA, ta có thể chọn $\lambda_0 = \lambda$, với hằng số $\lambda \le 2(1 - \sigma)/L$. Thuật toán GDA áp dụng được cho bước nhảy hằng số $\lambda \le 2(1 - \sigma)/L$.
            \pause
            \item  Nếu giá trị hằng số Lipschitz $L$ được biết trước, ta có thể chọn bước nhảy hằng số $\lambda \in (0, 2/L)$ như trong thuật toán gradient descent (GD) để giải các bài toán quy hoạch lồi.
        \end{itemize}
    \end{exampleblock}
    
\end{frame}


\subsection{2.2 Thuật toán Gradient Descent - GD}
\begin{frame}[shrink]{\textbf{\large 2.2.1 Thuật toán Gradient Descent - GD}}
\footnotesize
\begin{algorithm}[H]
\caption{(Gradient Descent Algorithm-GD)}
\begin{algorithmic}[1]

\State \textbf{Input: $x^0 \in C$; $\lambda_0 \in (0, 2/L)$; $k=0$ }
\State \textbf{Output: Stationary point $x^*$}

\While{true}
    \State $x^{k+1} \gets P_C\big(x^k - \lambda_k \nabla f(x^k)\big)$
    \If{$x^{k+1} = x^k$}
        \State \textbf{break}
    \EndIf

    \State $k \gets k + 1$
\EndWhile

\end{algorithmic}
\end{algorithm}
\end{frame}

\begin{frame}{\textbf{\large 2.2.2 Hệ quả}}
    \begin{exampleblock}{Hệ quả 1: }
        Giả sử $f$ lồi, $C = \mathbb{R}^m$ và $\{x^k\}$ là dãy được sinh ra bởi thuật toán GDA. Khi đó,
        $$f(x^k) - f(x^*) = O\left(\frac{1}{k}\right).$$
        với $x^*$ là nghiệm tối ưu của bài toán.
    \end{exampleblock}
\end{frame}


\begin{frame}{\textbf{\large 2.2.3 Chứng minh hệ quả 1}}
    \textbf{1. Thiết lập bài toán:}
    Gọi $x^*$ là nghiệm tối ưu. Đặt $\Delta_k := f(x^k) - f(x^*) \ge 0$.

    \textbf{2. Các đánh giá cơ bản:}
    Từ (\ref{eq:4}) kết hợp $x^k-x^{k+1} = \lambda_k \nabla \ f(x^k)$, ta có:
    \begin{equation} \label{eq:rate_1}
        \Delta_{k+1} \le \Delta_k - \sigma \lambda \|\nabla f(x^k)\|^2, \quad \forall k \ge k_1
    \end{equation}

    Do dãy $\{ x^k\}$ bị chặn và $f$ lồi nên :
    \begin{equation} \label{eq:rate_2}
    \begin{split}
        \Delta_k &\le \langle \nabla f(x^k), x^k - x^* \rangle \\
                 &\le \underbrace{\|x^k - x^*\|}_{\le M} \|\nabla f(x^k)\| \implies \|\nabla f(x^k)\| \ge \frac{\Delta_k}{M}
    \end{split}
\end{equation}
trong đó $M := \sup \{ \|x^k - x^*\| : k \ge k_1 \} < \infty$.
\end{frame}

\begin{frame}{\textbf{\large 2.2.3 Chứng minh hệ quả 1}}
    \textbf{3. Bất đẳng thức truy hồi:}
    
    Thay (\ref{eq:rate_2}) vào (\ref{eq:rate_1}), đặt $Q := \frac{\sigma \lambda}{M^2}$:
    \begin{equation} \label{eq:rate_3}
        \Delta_{k+1} \le \Delta_k - \sigma \lambda \frac{\Delta_k^2}{M^2} = \Delta_k - Q \Delta_k^2
    \end{equation}

    \textbf{4. Đánh giá tốc độ:}
    Vì $\Delta_{k+1} \le \Delta_k$, chia hai vế (\ref{eq:rate_3}) cho $\Delta_k \Delta_{k+1}$:
    \begin{align*}
        \frac{1}{\Delta_{k+1}} \ge \frac{1}{\Delta_k} + Q \frac{\Delta_k}{\Delta_{k+1}} \ge \frac{1}{\Delta_k} + Q 
        &\implies \frac{1}{\Delta_k} \ge \frac{1}{\Delta_{k_1}} + (k - k_1)Q \\
        &\implies \Delta_k \le \frac{1}{(k - k_1)Q}
    \end{align*}

    \textbf{Kết luận:} $$ f(x^k) - f(x^*) = O\left(\frac{1}{k}\right).$$
\end{frame}
\subsection{2.3 Biến thể ngẫu nhiên của thuật toán GDA - Stochastic GDA}
\begin{frame}{\textbf{\large 2.3.1 Biến thể ngẫu nhiên SGDA}}
    \textbf{1. Động lực}
    Ứng dụng trong Học sâu quy mô lớn (Large-scale Deep Learning) nơi việc tính toán toàn bộ gradient $\nabla f(x)$ là bất khả thi.

    \textbf{2. Bài toán Tối ưu Ngẫu nhiên:}
    Thay vì tối thiểu hóa hàm mục tiêu xác định, ta xét bài toán kỳ vọng:
    \begin{equation} \label{eq:stochastic_prob}
         \min_{x} \mathbb{E}[f_{\xi}(x)].
    \end{equation}
    Trong đó:
    \begin{itemize}
        \item $x$: Biến quyết định (tham số mô hình).
        \item $\xi$: Biến ngẫu nhiên với phân phối xác định.
        \item $f_{\xi}(x)$: Hàm $L$-smooth đối với mỗi $\xi$.
    \end{itemize}
\end{frame}


\begin{frame}{\textbf{\large 2.3.1 Biến thể ngẫu nhiên SGDA}}
    \textbf{3. Xấp xỉ Gradient}
    Tại mỗi bước lặp $k$, thay vì tính $\nabla \Phi(x^k) = \mathbb{E}[\nabla f_{\xi}(x^k)]$, ta lấy mẫu:
    \begin{itemize}
        \item Lấy mẫu ngẫu nhiên $\xi$ tại mỗi lần lặp $k$. 
        \item Tính gradient xấp xỉ: $\nabla f_{\xi^k}(x^k)$.
    \end{itemize}

    \textbf{4. Các thành phần trong Học máy:}
    \begin{table}[]
        \centering
        \begin{tabular}{c|l}
            \textbf{Ký hiệu} & \textbf{Ý nghĩa thực tế} \\
            \hline
            $x$ & Trọng số mạng nơ-ron  \\
            $\xi$ & Tính ngẫu nhiên của việc chọn dữ liệu \\
            $f_{\xi}(x)$ & Hàm mất mát trên một batch dữ liệu \\
            $\Phi(x)$ & Hàm mất mát tổng quát 
        \end{tabular}
    \end{table}
    
    \textit{*Lưu ý: Các phân tích lý thuyết chặt chẽ cho SGDA được dành cho nghiên cứu tương lai.}
\end{frame}

\begin{frame}[shrink]{\textbf{\large 2.3.2 Thuật toán Stochastic Gradient Descent (SGDA)}}
\footnotesize
\begin{algorithm}[H]
\caption{(Stochastic Gradient Descent Algorithm-SGDA)}
\begin{algorithmic}[1]

\State \textbf{Input: $x^0 \in C$; $\lambda_0 > 0$; $\sigma, \kappa \in (0,1)$; $k=0$ }
\State \textbf{Output: Stationary point $x^*$}
\While{true}
    \State Lấy mẫu ngẫu nhiên $\xi_k$

    \State $   x_{k+1} \gets P_C\big(x_k - \lambda_k \nabla f_{\xi_k}(x_k)\big)  $
    \If{$f_{\xi_k}(x_{k+1}) \le f_{\xi_k}(x_k)
        - \sigma \langle \nabla f_{\xi_k}(x_k), x_k - x_{k+1} \rangle$}
        \State
        $    \lambda_{k+1} \gets \lambda_k $
    \Else
        \State
        $
            \lambda_{k+1} \gets \kappa \lambda_k
        $
    \EndIf
    \If{$x_{k+1} = x_k$}
        \State \textbf{break}
    \EndIf
    \State  $k \gets k + 1$
\EndWhile
\end{algorithmic}
\end{algorithm}
\end{frame}





